\documentclass[11p, titlepage, oneside, a4paper]{article}
% Packages
\usepackage{amsmath}
\usepackage{graphicx}
\usepackage{hyperref}
\usepackage[english,swedish]{babel}
\usepackage[
    backend=biber,
    style=authoryear-ibid,
    sorting=ynt
]{biblatex}
\usepackage[utf8]{inputenc}
\usepackage[T1]{fontenc}
%Källor
\addbibresource{mall.bib}
\graphicspath{ {./images/} }

% Ändra de rader som behöver ändras
\def\inst{Teknikprogrammet}
\def\typeofdoc{Laborationsrapport}
\def\course{Fysik 1 150p}
\def\pretitle{Laboration 1}
\def\title{Rörelse: Hastighet och acceleration}
\def\name{Magnus Silverdal}
\def\username{magnus.silverdal}
\def\email{\username{}@ga.ntig.se}
\def\graders{Magnus Silverdal}

\begin{document}

\begin{titlepage}
		\thispagestyle{empty}
		\begin{large}
			\begin{tabular}{@{}p{\textwidth}@{}}
				\textbf{NTI gymnasiet \hfill \today} \\
				\textbf{\inst} \\
				\textbf{\typeofdoc} \\
			\end{tabular}
		\end{large}
		\vspace{10mm}
		\begin{center}
			\LARGE{\pretitle} \\
			\huge{\textbf{\course}}\\
			\vspace{10mm}
			\LARGE{\title} \\
			\vspace{15mm}
			\begin{large}
				\begin{tabular}{ll}
					\textbf{Namn} & \name \\
					\textbf{E-mail} & \texttt{\email} \\
				\end{tabular}
			\end{large}
			\vfill
            \includegraphics[width=0.5\textwidth]{images/NTI Gymnasiet_Symbol_print_svart.png}
			\vfill
            \large{\textbf{Handledare}}\\
			\mbox{\large{\graders}}
		\end{center}
	\end{titlepage}

    \begin{otherlanguage}{english}
	\begin{abstract}
        This is the ``sammanfattning'', it is written in english and called the abstract.      
    \end{abstract}
    \end{otherlanguage}
    % Om arbetet är långt har det en innehållsförteckning, annars kan den utelämnas
	\pagenumbering{roman}
	\tableofcontents
	
	% och lägger in en sidbrytning
	\newpage

	\pagenumbering{arabic}
	
	% i Sverige har vi normalt inget indrag vid nytt stycke
	\setlength{\parindent}{0pt}
	% men däremot lite mellanrum
	\setlength{\parskip}{10pt}
	
	\section{Syfte och frågeställning}
		Syftet med laborationen är att analysera rörelse för en vagn som rullar längs en bana och beräkna hastighet och acceleration under rörelsen.

	\section{Bakgrund och teori}
        Med utgångspunkt från en film av förloppet kan mjukvara för motion tracking utnyttjas för att få fram vagnens position vid olika tidpunkter. Denna information används sedan tillsammans med definitionerna av medelhastighet $v_m = \frac{\Delta s}{\Delta t}$ och medelacceleration $a_m = \frac{\Delta v}{\Delta t}$ för att beräkna ett approximativt värde för hastigheten och accelerationen som funktion av tiden. Med ett tillräckligt kort tidssteg så blir medelvärdet ungefär lika med momentanvärdet och i filmen är tidssteget som störst $\frac{1}{25}$ sekund.  \parencite{impuls}
	

	\section{Metod och materiel}
        \begin{enumerate}
            \item Vagn
            \item Lutande plan med ställning
            \item Linjal
            \item Mobilkamera
        \end{enumerate}
        
        Det lutande planet monteras på ställningen så att den ena änden är 1 dm över bordet, se figur \ref{fig:lutandeplan}. Linjalen placeras längs planet så att det finns en längdskala  i filmen. Kameran placeras vid sidan av uppställningen på ett avstånd så att hela rörelsen ryms i filmen utan att kameran behöver flyttas. Vagnen rullas nedför planet samtidigt som kameran filmar rörelsen. Försöket placeras så att ljusförhållanden och bakgrund ger en så tydlig och skarp film som möjligt.
        
        \begin{figure}[!h]
            \includegraphics[width=0.8\textwidth]{images/lutandePlan.jpg}
            \caption{En blid hade varit superbra här}
            \label{fig:lutandeplan}
        \end{figure}
        
        Filmen analyserades sedan med mjukvaran Tracker för att få fram en tabell med positionen som funktion av tiden.
    \newpage
	\section{Analys och beräkning}
        Datat från analysen av filmen visas i tabell \ref{table:result}
    
        
        \begin{table}
            \begin{center}
            \begin{tabular}{ |c|c| } 
                \hline
                Position (m) & Tid (s)  \\ 
                \hline
                0 & 0  \\ 
                0.1 & 0.02 \\
                \vdots & \vdots \\
                \hline
            \end{tabular}
                \caption{Mätvärden}
                \label{table:result}
            \end{center}
        \end{table}            
        

    Datat importeras i Excel och hastigheten beräknas med hjälp av formeln
    \begin{equation}
        v_m = \frac{\Delta s}{\Delta t}
    \end{equation}
    
    \section{Slutsats och resultat} 
        Resultatet av beräkningarna illustreras i graferna 2 och 3
    \section{Diskussion} 
    Resultatet är perfekt...

    
    \printbibliography

\end{document}

